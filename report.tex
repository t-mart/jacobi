\documentclass[12pt,letterpaper]{article}

\title{Report: Jacobi Algorithm}
\author{Tim Martin}
\date{Spring 2012}

\begin{document}

\maketitle

The Jacobi algorithm is an iterative method to find the eigenvalues of a matrix.
It works by selecting the largest absolute off-diagonal value, creating a Givens
rotation matrix, and applying it to the matrix. This causes the zero-ing of this
off-diagonal value, and when repeated several times, the resulting matrix will
be a diagonal matrix of only the eigenvalues.

The Jacobi method has a few caveats however. First, while zeroing an
off-diagonal value, other values in the matrix may change, including those the
algorithm has already zeroed. In a sense, this is negative progress, but
eventually, all the off-diagonals will converge towards zero.

Second, the value that you choose to zero should be the largest off-diagonal
value. Failure to do so results in a significantly slower convergence. The most
efficient off-diagonals are the largest because they have the greatest effect on
the diagonal property of the matrix.

\end{document}
